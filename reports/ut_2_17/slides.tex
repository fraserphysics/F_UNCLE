% Derived from projects/eos/trunk/ds13.tex.  Another good source might
% be projects/like9501/xcp8_1_14
\documentclass{beamer}
\setbeamertemplate{navigation symbols}{} %no nav symbols
\usepackage{amsmath,amsfonts}
\usepackage[pdftex]{rotating}
\newcommand\logx{y}
\newcommand\logf{g}
\newcommand\Logf{G}
\newcommand{\argmin}{\operatorname*{argmin}}
\newcommand{\La}{{\cal L}}
\newcommand{\C}{{\cal C}}
\newcommand{\normal}[2]{{\cal N}(#1,#2)}
\newcommand{\COST}{\cal C}
\newcommand{\LL}{{\cal L}}
\newcommand{\X}{{\cal X}}
\newcommand{\PS}{{\cal P}}
\newcommand{\Prob}{\text{Prob}}
\newcommand{\field}[1]{\mathbb{#1}}
\newcommand{\EV}[2]{\field{E}_{#1}\left[#2\right]}
\newcommand\inner[2]{\left<#1,#2\right>}
\newcommand{\nomf}{\tilde f}
\newcommand\Polytope[1]{\field{P}_{#1}}
\newcommand\RN{\field{R}^{N}}
\newcommand\PolytopeN{\Polytope{N}}
\newcommand{\T}{{\cal P}}
\newcommand{\partialfixed}[3]{\left. \frac{\partial #1}{\partial #2}\right|_#3}
\newcommand\CJ[1]{{{#1}_{\text{CJ}}}}
\newcommand\dum{\xi}
\newcommand\Ddum{d\dum}

\title{The F\_UNCLE Project: Functional UNcertainty Constrained by Law and Experiment\footnote{LA-UR-17-xxxx}}

\author{Andy Fraser}
\institute{Los Alamos National Laboratory}
\date{2017-2-8}

% \usetheme{Pittsburgh}
\usetheme{default}
\usefonttheme[]{serif}
\begin{document}
\frame{\titlepage
}

\frame{ \frametitle{Outline}\tableofcontents}

\section{Theory: EOS, Isentrope, Shock and Detonation}
\frame{
  \frametitle{Equations of State and Isentropes}

  \begin{columns}
    \begin{column}{0.6\textwidth}
      Picture here
      %\resizebox{0.99\textwidth}{!}{\includegraphics{invariant.pdf}}
    \end{column}
    \begin{column}{0.4\textwidth}
    \begin{align*}
      pv &= NRT & &\text{Ideal Gas} \\
      pv^3 &\approx c & &\text{Closer to my gas}
    \end{align*}
    \end{column}
  \end{columns}
}
\frame{
  \frametitle{Shocks, Conservation and Chapman Jouguet Conditions}

  \begin{columns}
    \begin{column}{0.4\textwidth}
      Assumptions:
      \begin{itemize}
      \item 1-d
      \item Instantaneous detonation
      \item Stationary in moving frame
      \item Conserve: Mass, momentum and energy
      \end{itemize}
    \end{column}
    \begin{column}{0.6\textwidth}
      Slope of \emph{Rayleigh line}: $-(\rho_0 D)^2$
      \begin{center}
        Picture
      \end{center}
    \end{column}
  \end{columns}
}

\section{Goals and Experiments}
\frame{
  \frametitle{Goals}
  %
  The stockpile stewardship mission involves simulating detonation
  experiments with multiple shocks.  For such simulations, we need a
  \emph{regional EOS}
  \begin{itemize}
  \item Near nominal \emph{CJ} isentrope
  \item Extension to higher densities
  \item Extension to higher entropies (temperatures)
  \end{itemize}
}
\frame{
  \frametitle{Use Data from Many Experiments}
  \begin{itemize}
  \item Rate sticks
  \item Overdriven flyers
  \item Copper clad cylinders
  \end{itemize}
}
\section{Analysis: Optimization and Uncertainty}
\frame{
  %
  \frametitle{Mie-Gruneisen form for Empirical EOS}
  %
  Fit two functions of specific volume along nominal \emph{CJ}
  isentrope:
  \begin{description}
  \item[$p(V)$] Pressure
  \item[$\Gamma = V\left( \frac {dp}{de} \right)_V$] Gruneisen gamma.
    Provides states off of the nominal \emph{CJ} isentrope.
  \end{description}
  From now on, I focus on $p(V)$.
}

\frame{
  \frametitle{Constrained Optimization}
  %
  \begin{description}
  \item[Constraints] As a function of volume, pressure must be
    \begin{itemize}
    \item Positive
    \item Monotonic
    \item \textbf{Convex}
    \end{itemize}
  \item[Cost] Weighted sum of squares
    $\leftrightarrow~log(\text{a posteriori probability})$ with
    Gaussian prior and likelihoods
    \begin{align*}
      C(\theta) &= -\frac{1}{2} \sum_i \left(x_i - \mu_i(\theta)
      \right)^T \Sigma_i^{-1} \left(x_i - \mu_i(\theta)
      \right) \\
      &\quad - \frac{1}{2} \left(\theta - \mu_\theta \right)^T
      \Sigma_\theta^{-1}  \left(\theta - \mu_\theta \right)
    \end{align*}

  \end{description}
}

\frame{
  \frametitle{Uncertainty and Fisher Information}
  After finding $\hat \theta \equiv \argmin_\theta C(\theta)$,
  characterize uncertainty by
  \begin{align*}
    \frac{\partial^2}{\partial \theta^2} C(\theta) &= \text{blah blah}
    \\
    &\quad \text{or} \\
    \EV{x}{\frac{\partial^2}{\partial \theta^2} C(\theta)} &=
          \text{blah} \\
  \end{align*}
  For each experiment,
  \begin{equation*}
    \cal{I}_i \equiv \text{blah}
  \end{equation*}
  is called the \emph{Fisher Information}
}

\frame{
  \frametitle{Fisher Information in a Surrogate Problem}
  Pictures
}

\frame{
  \frametitle{Software}
  We are using F\_UNCLE to learn and demonstrate \emph{Best Practices
    for Scientific Software}.  We use the following tools:
  \begin{itemize}
  \item Python, SciPy, NumPy, CVXOPT
  \item Git
  \item TravisCI
  \item Nosetests
  \item Sphinx
  \end{itemize}
}

\end{document}

\section{Uncertain $P(v)$ for Isentropic Expansion(Real Physics)}

\frame{ \frametitle{Isentropic Expansion}
  \begin{columns}
    \begin{column}[l]{0.55\textwidth}
      \resizebox{0.99\textwidth}{!}{\includegraphics{hixson.pdf}}\\
      Model from Hixson et al.\ JAP, 2000, \emph{Release isentropes of
        overdriven plastic-bonded explosive PBX-9501}
    \end{column}
    \begin{column}[r]{0.35\textwidth}
      \begin{itemize}
      \item $\pm \frac{1}{2} \%$
      \item Convex
      \item Monotonic
      \end{itemize}
     \bigskip

      Want probability measure $\mu$ for sets of functions to
      characterize uncertainty.
    \end{column}
  \end{columns}
 }

\section{Toy Problem: Uncertain $f(x)$}

\frame{ \frametitle{Toy Problem}
  \begin{columns}
    \begin{column}[l]{0.45\textwidth}
      \resizebox{1.0\columnwidth}{!}  {\input{schematic.pdf_t}}\bigskip
      
      \resizebox{0.9\textwidth}{!}{\includegraphics{lognom.pdf}}
    \end{column}
    \begin{column}[r]{0.45\textwidth}
      For ideal gun:\bigskip
      \begin{align*}
        \text{QOI}& 
        \begin{cases}
          T = \text{Muzzle time}\\
          E = \text{Muzzle energy}
        \end{cases}
      \end{align*}
      What is \emph{uncertainty}
      \begin{align*}
        \text{Given}&
        \begin{cases}
          f(x) = \frac{C}{x^3} \pm 2.5\%\\
          \text{Monotonic} \\
          \text{Convex}          
        \end{cases}
      \end{align*}
    \end{column}
  \end{columns}
}

\frame{ \frametitle{Allowed Functions at $E\times T$ Edge}
    \resizebox{0.7\textwidth}{!}{\includegraphics{pert.pdf}}
    \resizebox{0.7\textwidth}{!}{\includegraphics{allowedET.pdf}}
}

\section{Allowed Polytope, $\PolytopeN$, for $f(x) \, | \,x\in \left\{x_1,
    x_2, \ldots, x_N\right\}$}

\frame{ \frametitle{Allowed Polytope $\PolytopeN$}%
  If $\tilde f$ denotes the nominal function and
  $(x_1,x_2,\ldots,x_N)$ an ordered sequence of sample points.  Then
  the constraints on $f_i = f(x_i)$, namely
  \begin{align*}
    \Delta_i &\equiv x_{i+1} - x_i \\
    \epsilon_i &\equiv 0.025\cdot \tilde f(x_i) \\
    f_i & \leq \frac{\Delta_{i}f_{i-1} + \Delta_{i-1} f_{i+1} } {
    \Delta_{i-1} + \Delta_i } &&\text{\color{red}Convexity} \\
  f_i & \geq f_{i-1}          &&\text{\color{red}Monotonicity} \\
  f_i & \geq \nomf_i - \epsilon_i &&\text{\color{red}2.5\% error bound} \\
  \label{eq:constraint7}
  f_i & \leq \nomf_i + \epsilon_i &&\text{\color{red}2.5\% error bound}
\end{align*}
define an \emph{allowed polytope} $\textcolor{red}{\PolytopeN} \in
\RN$.  I want to characterize $\Polytope{\infty}$.  }

\section{Transitionally Invariant Metric, $L_1(g)$, for $g = \log(f)$}
\frame{
  \frametitle{Transitionally Invariant Log-log View}
  Choosing $(x_1,x_2,\ldots,x_N)$ uniformly on a log scale and
  expressing the constraints in log-log coordinates makes them shift
  invariant

  \begin{columns}
    \begin{column}[l]{0.75\textwidth}
      \resizebox{0.99\textwidth}{!}{\includegraphics{invariant.pdf}}
    \end{column}
    \begin{column}[l]{0.24\textwidth}
      New coordinates:
    \begin{align*}
      y &= \log(x) \\
      g &= \log\left(\frac{f}{\tilde f} \right)
    \end{align*}
    \end{column}
  \end{columns}
}

\frame{\frametitle{Allowed Region in 3-d}
  \vspace{-1.0cm}
  \begin{center}
    \resizebox{0.7\textwidth}{!}{\includegraphics[
      trim=0 0 0 -3cm]{allowed_tp2.pdf}}
  \end{center}\vspace{-1.5cm}
  Choose \textcolor{red}{$P(g_2|g_1,g_0)$} so that for every interior
  $\hat g$, the \textcolor{red}{epsilon ball} $\left\{ g: \int \left|
      g(y) - \hat g (y) \right| dy < \epsilon \right\}$ has the
  \textcolor{red}{same measure}, $\mu_\epsilon$.\bigskip
  
  Solve by quantizing $g$ and making all trajectories of length $N$
  have same probability.
}

\section{Uniform Probability from Eigenvectors of Adjacency Matrix}

\frame{ \frametitle{Transitions that Make all Trajectories
    Equiprobable}
  \begin{columns}
    \begin{column}[l]{0.45\textwidth}
      Choose $P_{X(\tau+1)|X(\tau)}$ to make
      \begin{equation}
        \label{eq:uniform}
        P(x_0^{L-1}) \approx e^{-Lh} ~\forall x_0^{L-1}
      \end{equation}
      (IE, achieve \textcolor{red}{\emph{entropy rate}} $h$ of
      \emph{sub-shift}.  See \textcolor{red}{Shannon} 1948, Theorem
      8.)\vspace{2mm}

      \resizebox{1.0\columnwidth}{!}  {\input{mt2.pdf_t}}

      Use the \textcolor{red}{\emph{cycle basis}} $(\alpha, \beta)$
      and the probabilities $(a, b, c)$
    \end{column}
    \begin{column}[r]{0.45\textwidth}
      \begin{align*}
        \alpha &= 0 \rightarrow 0 & P(\alpha) &=a \\
        \beta &= 0 \rightarrow 1 \rightarrow 0 & P(\beta)  &=bc
      \end{align*}
      EG. $010010 \mapsto \beta \alpha \beta$ and
      \begin{equation*}
        P(010010) = P_0 \left( P(\beta) \right)^2 P(\alpha)
      \end{equation*}
     Normalization and \eqref{eq:uniform}:
      \begin{align*}
        a^2 &= \left(P(\alpha)\right)^2 =  P(\beta) = bc \\
        a+b &= 1\\
        c &= 1
      \end{align*}
      Positive solution of system:
      \begin{equation*}
        \textcolor{magenta}{a = \frac{\sqrt{5}-1}{2}~~~ b =
            \frac{3-\sqrt{5}}{2}}
      \end{equation*}
    \end{column}
  \end{columns}
}

\frame{
  \frametitle{A Power Method}
  List all allowed trajectories of length
  $2T$, then estimate
  \begin{equation*}
    \hat \T_{i,j} =
    \frac{_{2T}N_{T,T+1}(i,j)}{_{2T}N_{T}(i)},\text{ where}
  \end{equation*}
  $_{2T}N_{T}(i)=$ number of length $2T$ trajectories for which
  $i$ is the value at position $T$\medskip

  $\textcolor{red}{_{2T}N_{T,T+1}(i,j)}=$ \textcolor{red}{number of
    trajectories} of length $2T$ that have values $i$ and $j$ at
  positions $T$ and $T+1$\bigskip

  The number of allowed sequences of length six that exactly
  matches a given sequence is either zero or one.  I write
  \begin{equation*}
    _6N_{1,2,3,4,5,6}(a,b,c,d,e,f) = A_{a,b}A_{b,c}A_{c,d}A_{d,e}A_{e,f}.
  \end{equation*}
  The number of sequences matching at positions three and four
  is
  \begin{equation*}
    \textcolor{red}{_6N_{3,4}(c,d)} =  \sum_{a,b,e,f}
    {_6N}_{1,2,3,4,5,6}(a,b,c,d,e,f) = \textcolor{red}{ 
    \sum_{a,f} \left(A^2\right)_{a,c} A_{c,d} \left(A^2\right)_{d,f}}
  \end{equation*}
}
\frame{
  \frametitle{A Power Method cont.}
  Similarly
  \begin{equation*}
    _{202}N_{101,102}(c,d) = \sum_{a,f} \left(A^{100}\right)_{a,c}
    A_{c,d} \left(A^{100}\right)_{d,f}
  \end{equation*}
  Defining $R(t)$ and $L(t)$ recursively as
  \begin{align*}
    L(1) &= \begin{bmatrix} 1,&1,&\cdots,&1,&1\end{bmatrix} \\
    N_L(t) &= \left| L(t) \right| \\
    L(t+1) &= \frac{L(t) A}{N_L(t)} \\
    R(1) &= L^{\text{T}}(1) \\
    N_R(t) &= \left| R(t) \right| \\
    R(t+1) &= \frac{A R(t)}{N_R(t)}.
  \end{align*}
  Note \textcolor{red}{$\lim_{t\rightarrow\infty}R(t) = \tilde R$ and
    $\lim_{t\rightarrow\infty}L(t) = \tilde L$} the right and left
  \textcolor{red}{eigenvectors of $A$} that correspond to the largest
  eigenvalue } \frame{ \frametitle{A Power Method cont.}  Using
  \begin{equation*}
    _{2(T+1)}N_{T+1,T+2}(c,d) \propto \left(L(T)\right)_c  A_{c,d}
    \left(R(T)\right)_d ,
  \end{equation*}
  I calculate the \textcolor{magenta}{maximum entropy} transition
  probabilities and the \textcolor{magenta}{stationary distribution}
  as follows
  \begin{align*}
    {\color{magenta}\tilde P_{i,j}} &= {\color{magenta}\tilde L_i A_{i,j}
      \tilde R_j} \\
    \tilde m_i &= \sum_j \tilde P_{i,j} \\
    P_{X(0)}(x) &= \frac{\tilde m_x}{\sum_i \tilde m_i} \\
    \T_{i,j} &= \frac{\tilde P_{i,j}}{\tilde m_i}
  \end{align*}
}

\frame{
  \frametitle{Spectral Decomposition of Covariance}
  \resizebox{0.99\textwidth}{!}{\includegraphics{PCA.pdf}}
}

\frame{
  \frametitle{Polytope Cross Sections}
  \resizebox{1.1\textwidth}{!}{\includegraphics{ellipsoid.pdf}}
}

\section{Continuum Limit?}
\frame{
  \frametitle{Conclusion}
  \begin{description}
  \item[Generality:] Uncertain functions are common.
  \item[Limitations:] My approach used translation invariance and
    compactness.
  \item[Incomplete:] I used a finite number of samples in $y$ and $g$.
    I want the \textcolor{magenta}{continuum limit}.
  \end{description}
}

\frame{
  \frametitle{Nominal Performance}
  \resizebox{0.5\textwidth}{!}{\includegraphics{nominal.pdf}}
}
\end{document}

%%%---------------
%%% Local Variables:
%%% eval: (TeX-PDF-mode)
%%% End:
