\documentclass[11pt]{article}
\usepackage{amsmath,amsfonts,afterpage}
%\usepackage{showlabels}

\title{Functional Uncertainty Constrained by Law and
  Experiment\footnote{LA-UR-16-21982}}

\author{Andrew M.\ Fraser\\Los Alamos National Laboratory}
\begin{document}
\maketitle
\begin{abstract}
  The F\_UNCLE project uses code to illustrate ideas for
  quantitatively characterizing uncertainty about functions.  The
  project is an evolving platform consisting of program and text files
  that numerically demonstrate and explore how either historical or
  proposed experiments have or could affect such uncertainty.  The
  poster focuses on a toy problem in which \emph{data} from a few
  experiments affect uncertainty about an unknown function that is
  constrained to be positive, monotonic and convex.  For each
  experiment, I use the \emph{Fisher Information} matrix, $J_k$, to
  quantify how the data constrains uncertainty about the unknown
  function.  A spectral decomposition of $J_k$ indicates the regions
  of the unknown function where the $k^{\text{th}}$ experiment is most
  constraining.  While neither physical measurements nor code for
  simulating physical experiments are distributed as part of F\_UNCLE,
  I use the ideas and some of the code for work on the equation of
  state of the gas produced by detonating explosives.  Highlights of
  that work appear at the end of the poster.
\end{abstract}

This document is a proposal to present a poster at SciPy2016 to be
held in Austin July 11-17.  (See http://scipy2016.scipy.org)  The
online form asks for:
\begin{enumerate}
\item A title
\item An abstract of less than 500 words
\item A long description for evaluating the proposal
\end{enumerate}
Once I get an LAUR for this document, I will submit those portions of
this document.

\section{Long Description}
\label{sec:long}

A need to quantitatively characterize the following three notions
drives this work: 1. Uncertainty about functions in physical
theories. 2. The manner in which existing experimental data constrain
that uncertainty.  3. The value that constraints from proposed
experiments would provide.  I chose Python and SciPy to make the
analysis coherent, reproducible and extendable.

I have two versions of this project: 1. Code that runs in a minute on
a desktop computer; 2. Code that runs for hours on high performance
computers.  The two versions share the same ideas and some code.  Both
use Python and SciPy to organize sequential batches of simulations in
a search for the \emph{Maximum A posteriori Probability} function for a
specific prior and a collection of experimental data.  After finding
the MAP function, the code uses the second derivative of the
log-likelihood for data from each experiment to estimate the
corresponding Fisher Information matrix.  The current version of the
code nicely illustrates that the information from different
experiments constrain different parts of the unknown function.

I am designing the code to have a framework that will simultaneously
handle several uncertain functions and several experiments.  The
framework code will be the same for all problems and platforms.
Separate modules will implement each function and experiment.

I expect permission to post code to https://github.com/fraserphysics
before the conference.
\end{document}

%%%---------------
%%% Local Variables:
%%% eval: (TeX-PDF-mode)
%%% eval: (setq ispell-personal-dictionary "./localdict")
%%% End:
