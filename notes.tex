\documentclass[11pt]{article}
\usepackage{amsmath,amsfonts,afterpage}
%\usepackage{showlabels}
\usepackage[pdftex]{graphicx,color}
\usepackage{url}
\usepackage{bigfoot}
\newcommand{\normal}[2]{{\cal N}\left( #1,#2 \right)}
\newcommand{\normalexp}[3]{ -\frac{1}{2}
      (#1 - #2)^T #3^{-1} (#1 - #2) }
\newcommand{\La}{{\cal L}}
\newcommand{\fnom}{\tilde f}
\newcommand{\fhat}{\hat f}
\newcommand{\COST}{\cal C}
\newcommand{\LL}{{\cal L}}
\newcommand{\Fisher}{{\cal I}}
\newcommand{\Prob}{\text{Prob}}
\newcommand{\field}[1]{\mathbb{#1}}
\newcommand\REAL{\field{R}}
\newcommand\Z{\field{Z}}
\newcommand{\partialfixed}[3]{\left. \frac{\partial #1}{\partial
      #2}\right|_#3}
\newcommand{\partiald}[2]{\frac{\partial #1}{\partial #2}}
\newcommand{\argmin}{\operatorname*{argmin}}
\newcommand{\argmax}{\operatorname*{argmax}}
\newcommand\norm[1]{\left|#1\right|}
\newcommand\bv{\mathbf{v}}
\newcommand\bt{\mathbf{t}}
\newcommand\vol{v}        % specific volume
\newcommand{\pressure}{p}
\newcommand{\eos}{f}
\newcommand{\eosnom}{\tilde f}
\newcommand{\EOS}{{\cal F}}
\newcommand{\data}{y}
\newcommand{\DATA}{{\cal Y}}
\newcommand\Vfunc{\mathbb{V}}
\newcommand\Vt{\mathbf{V}}
\newcommand\vexp{V_{\rm{exp}}}
\newcommand\texp{T_{\rm{exp}}}
\newcommand\cf{c_f}
\newcommand\cv{c_v}
\newcommand\fbasis{b_f}
\newcommand\vbasis{b_v}
\newcommand\tsim{{\mathbf t}_{\rm{sim}}}
\newcommand\DVDf{\partiald{\Vt}{f}}
\newcommand\Lbb{\mathbb{L}}
\newcommand\epv{\epsilon_v}
\newcommand\epf{\epsilon_f}
\newcommand{\fig}[1]{figs/#1.pdf}

\title{Notes on EOS Estimation Code}

\author{Andrew M.\ Fraser}
\begin{document}
\maketitle
\begin{abstract}
  The source for these notes and illustrations reside in
  like\_eos/notes.tex\footnote{To obtain the source, build the
    document and view it, run the following sequence of commands:
\begin{verbatim}
->git clone ssh://git@xcp-stash.lanl.gov:7999/fvs/like_eos.git
->cd like_eos/
->make -j8
->evince notes.pdf 
\end{verbatim}
}.
  I use the same ideas for work on the 9501 model on moonlight.
\end{abstract}

\section{Introduction}
\label{sec:introduction}

I describe theory and code for numerical experiments based on functions
(EOSs) that map specific volume to pressure along an isentrope.
Running the code helps verify the theory.  I intend to describe how
tightly a few different experiments constrain the EOS.  This document
and the code are structured as follows:
\begin{description}
\item[EOS:] I characterize the EOS $f$ in terms of
  \begin{description}
  \item[Constraints:] The function $f$ must be positive, monotonic and
    convex over its entire domain.
  \item[Prior:] I use a Gaussian distribution over parameters of an ad
    hoc choice of coordinates.  I don't know how seriously this flawed
    choice affects the results.
  \end{description}
\item[Experiments:] I will consider a set of experiments
  $\left\{ E_k: 0 \leq k < N \right\}$.  For each experiment, I will
  need:
  \begin{description}
  \item[Data:] Theory and code for generating simulated data
  \item[Likelihood:] Theory and code for calculating the likelihood of
    data
  \end{description}
\item[Analysis:] If the likelihoods were exponentials of quadratic
  functions of the parameters of the EOS, then the a posteriori
  distribution would be Gaussian if there were no constraints.  The
  code for each experiment reports the dependence of the data on the
  parameters as an affine approximation.  That lets me approximately
  characterize the likelihoods as exponentials of quadratics.  I
  enforce the constraints and ignore their effects on the
  distributions.  The file \emph{fit.py} implements the resulting
  constrained MAP estimation.
\end{description}

\section{The EOS}
\label{sec:eos}

The class \emph{Spline\_eos} in the file \emph{eos.py} provides
methods that implement the EOSs described in this section.

\subsection{Definitions and Notation}
\label{sec:eos-notation}

\begin{description}
\item[$\vol$:] Specific volume
\item[$\pressure$:] Pressure
\item[$\eos$:] An EOS that maps specific volume to pressure, $\eos: \vol
  \mapsto \pressure$.
\item[$\EOS$:] A set of possible EOS functions.
\item[$\cf,\fbasis$:] Vector of coefficients and cubic spline basis
  functions that define an EOS.  I will use $\cf[i]$ and $\fbasis[i]$
  to denote components.
\item[$\mu_\eos, \Sigma_\eos$:] Mean and covariance of prior
  distribution of EOS.  In a context that requires coordinates, I let
  $\mu_\eos = \left( \cf[0], \cf[1], \ldots , \cf[n] \right)^T$.
\end{description}

\subsection{An Inconsistently Defined Prior}
\label{sec:eos-prior}

I constrain $\eos$ to be positive and to be a convex function of
$\vol$.  I have begun but not finished work to define a probability
measure on a set of functions that obeys those constraints and is
further constrained by $\frac{\left| \eos(\vol) - \mu_\eos(\vol)
  \right|}{\mu_\eos(\vol)} \leq \Delta$.  For now, I characterize the
prior as Gaussian.  However as I search for the mean of the a
posteriori distribution, I enforce the constraints.  This is
inconsistent and requires justification or a different approach.

I use a prior defined in terms of spline coefficients.  I start with a
nominal EOS
\begin{equation}
  \label{eq:2}
  \eosnom(\vol) = \frac{C}{\vol^3}, \text{ where } C \leftrightarrow
  2.56\times10^9 \text{Pa} \text{ at one gram/cc}
\end{equation}
and over a finite domain I approximate it by a cubic spline with
coefficients $\left\{\tilde \cf[i] \right\}$.  Then I assign a variance to
  each coefficient:
\begin{equation}
  \label{eq:3}
  \sigma^2[i] = \left( \cf[i] \Delta \right)^2.
\end{equation}
To match Shaw's $\frac{1}{2} \%$ claim, I set $\Delta = 0.005$.  These
choices yield:
\begin{align*}
  \mu_\eos &\leftrightarrow \left\{\tilde c[i] \right\} \\
  \Sigma_\eos[i,j] &= \tilde \sigma^2[i] \delta_{i,j}
\end{align*}

\subsection{The Nominal and \emph{True} EOS}
\label{sec:true-eos}


\newcommand{\freq}{k} %
For each experiment, data comes from a simulation using the
\emph{true} function
\begin{align}
  \label{eq:actual}
  f(v) &= \fnom(v) + \frac{C}{v_0^3} e^{-\frac{(v-v_0)^2}{2w^2}}\\
   \text{ where }C &= 2.56\times 10^{10} \nonumber \\
  w &= 0.062. \nonumber \\
  v_0 &= 0.35. \nonumber
\end{align}
The EOS specified by \eqref{eq:actual} is like the prior except that
there is an added Gaussian perturbation centered at $v=v_0$.

\section{Experiments}
\label{sec:experiments}

I've written the code with a structure that lets me include more
experiments easily.  Each experiment must have theory and code that
provides faux data, Gaussian likelihood as a function of the EOS
$\eos$ and the derivative of the conditional mean with respect to the
EOS.  For the $k^{\rm th}$ experiment, here is the notation:
\begin{description}
\item[$\data_k\in \DATA_k$:] Faux data and the set over which the
  likelihood is a density.
\item[$\mu_k$:] A function $\EOS\mapsto \DATA_k$ that gives the mean
  of the likelihood.
\item[$\Sigma_k$:] The covariance of the likelihood.  This is
  independent of $\eos$.
\item[$D_k$:] The derivative of $\mu_k$ with respect to $\eos$, which
  in the spline coordinates is
  $D_k[i,j] =\frac{\partial \data_k[i]}{\partial \cf[j]}$.
\item[$\epsilon_k$:] The difference between the data and the model
  prediction, $\epsilon_k = \data_k - \mu_k(\eos)$.
\item[log likelihood:] Ignoring constant terms that depend on
  $\Sigma_k$, the log likelihood is $-\frac{1}{2}\epsilon_k^T
  \Sigma_k^{-1} \epsilon_k$.
\end{description}

\subsection{Fisher Information}
\label{sec:fisher-information}

Fisher Information describes how tightly an experiment constrains
unknown parameters.  Assume a parametric probability density function
$p(Y|\theta)$ that gives the likelihood for data $Y$.  Quoting
Wikipedia: ``If $p(y|\theta)$ is twice differentiable with respect to
$\theta$, and under certain regularity conditions, then the Fisher
information may also be written as''
\begin{equation*}
  \mathcal{I}(\theta) = - \operatorname{E}
  \left[\left. \frac{\partial^2}{\partial\theta^2} \log
      p(Y;\theta)\right|\theta \right].
\end{equation*}
In my notation for experiments and models, the parameters are $c$.
Dropping added constants, the log likelihood is
\begin{equation*}
  \log(p(y|c)) + C = -\frac{1}{2}\epsilon_k^T \Sigma_k^{-1} \epsilon_k
  = -\frac{1}{2}(y-\mu(c))^T \Sigma_k^{-1} (y-\mu(c)).
\end{equation*}
At the maximum, the first derivative is zero and I calculate
\newcommand{\dmudc}{\left(\frac{\partial \mu(c)}{\partial c} \right)}
\begin{align*}
  \frac{\partial}{\partial c} \log(p(y|c)) &= 
      (y-\mu(c))^T \Sigma_k^{-1} \dmudc \\
  \frac{\partial^2}{\partial c^2} \log(p(y|c)) &=
      \dmudc^T \Sigma_k^{-1} \dmudc +  (y-\mu(c))^T \Sigma_k^{-1}
       \left(\frac{\partial^2 \mu(c)}{\partial c^2} \right).
\end{align*}
Since the distribution of $Y$ is symmetric about its mean $\mu(c)$,
the expected value of the second term is 0, and the first term is the
Fisher Information.

\subsection{A Rate Stick}
\label{sec:rate-stick}

Here the data are a sequence of times that a detonation shock is
measured arriving at locations along a rate stick that is so thick
that the detonation velocity is not reduced by curvature.  The code
for the faux data uses the average density and sensor positions given
by Pemberton et al.\cite{Pemberton9501} for their \emph{Shot 1}.

\subsubsection{Detonation Velocity}
\label{sec:det-vel}

The only characteristic that this ideal rate stick measures is the
detonation velocity.  Code in \emph{stick.py} derives that velocity
following Section 2A (entitled \emph{The Simplest Theory}) of
\emph{Detonation Theory and Experiment} by Fickett and
Davis\cite{FickettDavis}.  At the Chapman Jouguet (CJ) state, the
following three curves are tangent in the $p,v$ plane:
\begin{itemize}
\item The Rayleigh line which gives a relation implied by conservation
  laws between pressure and density (or specific volume) before and
  after a shock.
\item The Hugoniot curve, which I will not use.
\item An isentrope.  Generally one must use the Hugoniot to determine
  which isentrope goes through the CJ state, but I will just assume
  that each isentrope considered goes through the CJ state.
\end{itemize}

On page 17 of Fickett and Davis, Equation 2.3 expresses the Rayleigh
line as
\begin{equation}
  \label{eq:Rayleigh}
  \rho_0^2 V^2 - (p-p_0)/(v_0-v) = 0,
\end{equation}
with the following notation:
\begin{description}
\item[$\rho_0$] The initial density (before detonation wave arrives)
\item[$v_0\equiv\frac{1}{\rho_0}$] The initial specific volume
\item[$p_0$] The initial pressure
\item[$V$] The velocity of the detonation wave
\item[$p$] The pressure at positions behind the wave
\item[$v$] The specific at positions behind the wave.
\end{description}
Rearranging the terms in \eqref{eq:Rayleigh} yields this relation
between pressure and volume after the shock
\begin{equation*}
  p = R(v,V) \equiv p_0 + \frac{V^2(v_0-v)}{v_0^2}.
\end{equation*}
At the CJ point, the Rayleigh line is tangent to the isentrope.
Using
\begin{align*}
  F(v,V) &= \eos(v) - R(v,V) \\
  F'(v,V) &= \frac{d \eos}{d v} - \frac{V^2}{v_0^2},
\end{align*}
I can write the following conditions for the CJ point:
\begin{subequations}
  \begin{align}
    \label{eq:Fcond}
    F(v,V) &= 0 \\
    \label{eq:dFcond}
    F'(v,V) &= 0.
  \end{align}
\end{subequations}
Given $V$, the code has function that numerically solves
\eqref{eq:dFcond} for $v$.  Letting $v(V)$ denote that solution, I
write \eqref{eq:Fcond} as
\begin{equation}
  \label{eq:FV}
  F(v(V),V) = 0.
\end{equation}
The code numerically solves \eqref{eq:FV} to get $V_{\text{CJ}}$ and
then assigns $v_{\text{CJ}} = v(V_{\text{CJ}})$.
Figure~\ref{fig:CJ_stick} depicts three isentropes and the results of
solving \eqref{eq:FV} for the two labeled \emph{experiment} and
\emph{fit}.

\begin{figure}
  \centering
    \resizebox{\columnwidth}{!}{\includegraphics{\fig{CJ_stick}}}
    \caption{Isentropes, Rayleigh lines and CJ conditions. Starting
      from the isentrope labeled \emph{nominal} and using data from a
      simulated experiment based on the isentrope labeled
      \emph{experiment}, the optimization algorithm described in
      Section~\ref{sec:algorithm} produced the estimate labeled
      \emph{fit}.  Solving Eqn.~\eqref{eq:FV} for the
      \emph{experiment} and \emph{fit} isentropes yields the two
      Rayleigh lines that appear.  They are are nearly identical
      because the detonation velocities (and hence the experimental
      and fit data) are given by their slopes.  Outside of the CJ
      points where the Rayleigh lines are tangent to the isentropes,
      the data does not constrain the isentropes, and in fact they are
      quite different.}
  \label{fig:CJ_stick}
\end{figure}

\subsubsection{Simulating the Measured Data}
\label{sec:stick-sim}

From $V_{\text{CJ}}(\eos)$, the code calculates
\begin{equation}
  \label{eq:t}
  t[j] = \frac{x[j]}{V_{\text{CJ}}(\eos)}.
\end{equation}
Finally the code calculates
\begin{equation}
  \label{eq:Dstick}
  D[j,i] \equiv \frac{\partial t[j]}{\partial c[i]}
\end{equation}
using finite differences.  Illustrations of simulations appear in
Figure~\ref{fig:opt_stick}.

\begin{figure}
  \centering
    \resizebox{\columnwidth}{!}{\includegraphics{\fig{opt_stick}}}
    \caption{Fitting an isentrope to rate stick data.  In the upper
      plot, black +'s denote measured shock arrival time at 7
      positions.  The blue line represents the shock velocity
      calculated from the nominal EOS, and the other lines come from
      the sequence of isentropes that the optimization algorithm described in
      Section~\ref{sec:algorithm} generates as it seeks an isentrope
      that will produce a simulation that matches the data.  That
      sequence of isentropes appears in the lower plot.}
  \label{fig:opt_stick}
\end{figure}

\subsubsection{Fisher Information Matrix for the Rate Stick}
\label{sec:stick-information}

As Fig.~\ref{fig:CJ_stick} indicates, only the CJ point on the EOS
influences the forecast data, $\mu(c)$.  Thus only one degree of
freedom in the model influences the likelihood and Fisher Information
matrix, $\Fisher$ should have a rank of one.
Figure~\ref{fig:info_stick} illustrates characteristics of the
optimization procedure and $\Fisher(\hat c)$.  The largest eigenvalue
$\Fisher(\hat c)$ is $10^{16}$ larger than the next largest, ie, the
rank of $\Fisher(\hat c)$ is one to within machine
precision.\marginpar{Interpret/explain $\Fisher$ eigenfunction as
  constraint.}

\begin{figure}
  \centering
    \resizebox{\columnwidth}{!}{\includegraphics{\fig{info_stick}}}
    \caption{Fisher Information of the Rate Stick Experiment.  The
      sequence of log likelihoods produced by the optimization
      procedure appear in the upper left, and the corresponding
      isentropes appear in the upper right.  The largest three
      eigenvalues of $\Fisher(\hat c)$ appear in the lower left and
      the eigenfunction corresponding to the largest eigenvalue
      appears in the lower left.}
  \label{fig:info_stick}
\end{figure}

\subsection{The Gun}
\label{sec:gun}

The data is a time series of measurements of the velocity of a
projectile as it accelerates down a gun barrel.

\subsubsection{Notation}
\label{sec:basic_notation}

\begin{description}
\item[$y\equiv(\vexp, \texp)$:] An experimental sequence of measured pairs of
  velocity and time
\item[$\bv(\eos)$:] A sequence of model velocities
\item[$\Vt$:] A map from times to model velocities, eg, $\bv =
  \Vt(\texp)$
\item[$\tsim$:] A sequence of closely spaced sample times at which to record
  simulated position and velocity for constructing $\Vt$.
\item[$\Vfunc$:] An expensive procedure that maps an EOS function to a
  function that maps time to velocity with $\Vt = \Vfunc(\eos)$
\item[$\cv,\vbasis$:] Vectors of spline coefficients and basis functions
  that define a $\Vt$
\item[$C$] Derivative of the velocity function with respect to the
  EOS, ie, matrix with elements $C[k,i] = \partiald{\cv[k]}{\cf[i]}$
\item[$B$] Matrix with elements $B[j,k] = \vbasis[k](\texp[j])$, ie,
  the exactly linear map from $c_v$ to simulated data.  The matrix
  elements are the velocity basis functions evaluated at the
  experimental sample times.  Thus the derivative of the simulated
  data with respect to the EOS basis coefficients is $D = BC$.
\item[$\epv$] The vector of differences between the measured
  velocities and the simulated velocities at the times of the
  measurements.
\end{description}

\subsubsection{Implementation}
\label{sec:basic_implementation}

\begin{description}
\item[$\Vfunc(f)$:] Run a simulation and record a sequence of
  velocities $\bv$ at times $\tsim$.  Then fit a
  spline to $(\bv, \bt)$ to obtain $\Vt\iff \{\cv,\vbasis\}$.
\item[$\Vt$:] Call the spline evaluation method to get $\bv = \Vt(\texp)$
\item[$D$:] Evaluate $\Vfunc(f)$ for $N(\cf)+1$ different sets of
  coefficients $\cf$, and use finite differences to get a matrix with
  elements $C[k,i] = \partiald{\cv[k]}{\cf[i]}$.  Then
  \begin{equation*}
    D[j,i] = \partiald{\Vt[j]}{\cf[i]} = \sum_k C[k,i] \vbasis[k](T) = BC[j,i]
  \end{equation*}
\end{description}
Figure~\ref{fig:big_d} illustrates the calculation of $D$ using finite
differences.
\begin{figure}
  \centering
    \resizebox{\columnwidth}{!}{\includegraphics{\fig{big_d}}}
    \caption{Illustration of the derivative of the velocity function
      with respect to the pressure function, $\frac{d v}{d p}$.  The
      second and third columns correspond variations of the
      coefficients of basis functions that differ by a factor of ten.
      The lower left plot indicates that the nonlinear part of the
      response is a factor of a $10^4$ smaller than the linear
      response.  }
  \label{fig:big_d}
\end{figure}

\subsubsection{The Code}
\label{sec:guncode}

The file \emph{gun.py} defines the \emph{Gun} class which has the
following methods:
\begin{description}
\item[fit\_t2v] Run a simulation to build a spline for mapping times
  to velocities.  The result for the initial EOS appears in
  Fig.~\ref{fig:vt_gun}.
  \begin{figure}
    \centering
    \resizebox{\columnwidth}{!}{\includegraphics{\fig{vt_gun}}}
    \caption{Velocities as functions of time.  Traces for the
      \emph{experimental} data, a simulation with the nominal EOS,
      $\mu_\eos$, and the error, $\epsilon$.}
    \label{fig:vt_gun}
  \end{figure}
\item[fit\_C] Use finite differences to calculate dv/df in terms of
  spline coefficients, ie,
  $C[k,i] = \frac{\Delta c_v[k]}{\Delta c_f[i]}$.  See
  Fig.~\ref{fig:C_gun} and note that $k$ and $i$ range over components
  of $\cv$ and $\cf$ respectively.
  \begin{figure}
    \centering
    \resizebox{\columnwidth}{!}{\includegraphics{\fig{C_gun}}}
    \caption{The matrix $C$.  For each $i$, a trace represents
      $C[k,i] = \frac{\partial \cv[k]}{\partial \cf[i]}$.}
    \label{fig:C_gun}
  \end{figure}
\item[fit\_B\_ep] Map experimental velocities v and times t to the
  following:
  \begin{align*}        
    \epsilon[i] &= v[j] - t2v(t[j])
                        \text{ Difference between simulation and data} \\
    B[j,k] &= \vbasis[k](t[j])     
  \end{align*}
  Notes: $\vbasis[k]$ is the $k^{\text{th}}$ basis function for the t2v spline.
  $B$ is the derivative of the simulated velocity function at the
  experimental sample times wrt the coefficients of the spline fit
  that implements that function, and $BC_{j,i}$ is the derivative of
  $v(t[j])$ wrt $\cf[i]$.  See Fig.~\ref{fig:BC_gun}.
  \begin{figure}
    \centering
    \resizebox{\columnwidth}{!}{\includegraphics{\fig{BC_gun}}}
    \caption{The product $D = B C$.  For each $i$, a trace illustrates
      $D[j,i] = \frac{\partial v(t[j])}{\partial c_f[i]}$, and it
      represents a difference between two splines fit to simulations
      with a finite difference in $f$.  Because the basis functions
      for $v(t)$ are very narrow, this figure is very similar to
      Fig.~\ref{fig:C_gun}.}
    \label{fig:BC_gun}
  \end{figure}
\end{description}

\begin{figure}
  \centering
    \resizebox{\columnwidth}{!}{\includegraphics{\fig{info_gun}}}
    \caption{Fisher Information of the Gun Experiment.  The sequence
      of log likelihoods produced by the optimization procedure appear
      in the upper left, and the corresponding isentropes appear in
      the upper right.  The largest nine eigenvalues
      of$\Fisher(\hat c)$ appear in the lower left and the
      eigenfunction corresponding to the largest four eigenvalue
      appears in the lower left.}
  \label{fig:info_gun}
\end{figure}

\subsection{A Cylinder}
\label{sec:cylinder}

\marginpar{Not implemented yet}
This is a simplification of measurements of cylinders.  Suppose that
all of the material in a long cylinder detonates in an instant and
that the density after that is uniform inside.  Thus the radius as
function of time, $\left\{r(t): t>0 \right\}$, is a sufficient
specification of an experimental result.  Letting $r(0) = r_0$ denote
the initial radius, the work done by the gas in expanding to $r$ is
\begin{equation*}
  U(r) = \int_{r_0}^r p(v(r)) \frac{dv}{dr} dr = \int_{r_0}^r p(\pi
  r^2) 2\pi r dr.
\end{equation*}
That work is used to deform and accelerate the cladding.  I model the
energy of deformation is
\begin{equation*}
  D(r) = \sigma \log \left(\frac{r}{r_0} \right),
\end{equation*}
and the kinetic energy of the cladding is
\begin{equation*}
  K = \frac{1}{2} m \left( \dot r \right)^2.
\end{equation*}
Thus the velocity as a function of radius is
\begin{equation*}
  v(r) = \sqrt{\frac{2}{m} \left( U(r) - \sigma \log
      \left(\frac{r}{r_0} \right) \right) },
\end{equation*}
and I can obtain the velocity as a function of time by numerical
integration.

\section{Analysis}
\label{sec:analysis}

To characterize the information about the EOS that a collection of
measurements provide, I use a Bayesian analyses that assumes all the
relevant distributions are Gaussian.  The analysis is flawed because
the EOS is constrained in a manner that is inconsistent with the
Gaussian assumption.

In Section~\ref{sec:eos} I described a prior with the form
\begin{equation*}
  \eos \sim \normal{\mu_\eos}{\Sigma_\eos},
\end{equation*}
and in Section~\ref{sec:experiments} I suggested the form
\begin{equation*}
  \data | \eos \sim \prod_k \normal{\mu_k(f)}{\Sigma_k}
\end{equation*}
for the likelihood.  The a posteriori distribution is
\begin{equation*}
  p(\eos|\data) = \frac{p(\data|\eos) p(\eos)}{\int p(\data|\eos)
    p(\eos) d \eos}.
\end{equation*}
Without approximation, the assumptions so far imply
\begin{equation}
  \label{eq:app}
  \log(p(\eos|\data)) = \normalexp{\data}{\mu_\data(\eos)}{\Sigma_\data}
\normalexp{\eos}{\mu_\eos)}{\Sigma_\eos} + C,
\end{equation}
where the constant $C$ does not depend on $f$.
Equation~\eqref{eq:app} is not a Gaussian distribution of $\eos$
unless $\mu_\data(\eos)$ is affine, ie,
$\mu_\data(\eos) = \mu_\data(\hat \eos) + D(\eos - \hat \eos)$, in
which case
\begin{equation*}
  -2\log(p(\eos|\data)) = \left(\eos - \hat \eos \right)^T
  \left( D^T \Sigma_\data^{-1} D + \Sigma_\eos^{-1} \right)
  \left(\eos - \hat \eos \right) + C,
\end{equation*}
where $\hat \eos$ maximizes \eqref{eq:app}.  I will assume that near
$\hat \eos$, ie, the MAP estimate, an affine approximation is adequate
and write
\begin{equation}
  \label{eq:normal_app}
  \eos|\data \sim \normal{\hat \eos}{\left( D^T \Sigma_\data^{-1} D + \Sigma_\eos^{-1} \right)^{-1}},
\end{equation}
where
\begin{equation*}
  D \equiv \left. \frac{\partial\mu_\data}{\partial \eos}
  \right|_{\hat \eos} .
\end{equation*}
Both the optimization procedure that I use to search for $\hat \eos$
and my characterization of the a posteriori distribution of $\eos$
rely on having access to $D$.  Note that
\begin{equation}
  \label{eq:Fisher_I}
   D^T \Sigma_\data^{-1} D = \sum_k D_k^T \Sigma_k^{-1} D_k,
\end{equation}
and $D_k^T \Sigma_k^{-1} D_k$ approximates the second derivative of
the log likelihood which is called the \emph{Fisher Information} of
the $k^{\text{th}}$ experiment.

\subsection{An Algorithm}
\label{sec:algorithm}

\begin{enumerate}
\item \label{step:0} Initialize $c_0 = \tilde c$, where this vector of
  spline coefficients represents $\tilde \eos$.
\item \label{step:1} For iteration $i$ start with $c_i$ and
  calculate
  \begin{description}
  \item[$D=\cup_k \left\{ D_k \right\},$] the derivative of the model
    predictions of the data for all the experiments.  For the
    $k^{\text{th}}$ experiment
    \begin{equation*}
      D_k \equiv \left. \frac{\partial\mu_k}{\partial c}
      \right|_{c_i} .
    \end{equation*}
  \item[$\epsilon=\cup_k \left\{ \epsilon_k \right\},$] the difference
    between the model predictions and the experimental data.  For the
    $k^{\text{th}}$ experiment
    \begin{equation*}
      \epsilon_k \equiv \data_k - \mu_k(c_i) .
    \end{equation*}
  \end{description}
\item \label{step:2} Calculate $c_{i+1} = c_i + d$ by solving
  the quadratic program
\begin{subequations}
  \label{eq:dhat}
  \begin{align}
    \text{Minimize } & \frac{1}{2} d^T P d + q^T d \\
  \label{eq:dhatb}
    \text{Subject to } & Gd \preceq h
  \end{align}
\end{subequations}
with $G$ describing the constraints (see section
\ref{sec:constraints}) and
\begin{subequations}
  \label{eq:opt}
  \begin{align}
    P &= \Sigma_\eos^{-1} + \sum_k D_k^T\Sigma_k^{-1} D_k \\
    q^T &= (c_i - c_0)^T \Sigma_\eos^{-1} - \sum_k \epsilon_k
          \Sigma_k^{-1} D_k \\
    h &= -Gc_i
  \end{align}
\end{subequations}
\item \label{step:3} Calculate $L_i = \log(p(c_{i+1}|\data)$ using
  \eqref{eq:app}.
\item \label{step:4} Stop if $\left( L_0, L_1, \ldots L_{i+1} \right)$
  meets some convergence criterion, otherwise go back to \ref{step:1}.
\end{enumerate}

\subsection{Numerical Results}
\label{sec:numerical-results}

The result of a single iteration of the algorithm appears in
Fig.~\ref{fig:opt_result}.  Three iterations of the algorithm
increases $L_i$ from $-9.7\times 10^8$ to $-4.2\times 10^4$.
Figure~\ref{fig:fve_gun} illustrates the convergence.
\begin{figure}
  \centering
  \resizebox{\columnwidth}{!}{\includegraphics{\fig{opt_result}}}
  \caption{The function $f_1 = f_0 + \hat d$ returned by fit.opt()
    compared to $f_0$.}
  \label{fig:opt_result}
\end{figure}
\begin{figure}
  \centering
    \resizebox{\columnwidth}{!}{\includegraphics{\fig{fve_gun}}}
    \caption{Sequential estimation of the maximum a posteriori
      probability parameters of $f$.  The \emph{true} EOS appears as
      \emph{experimental} in the upper plot, and the optimization starts
      with the \emph{nominal} and ends with \emph{fit}.  The
      corresponding velocity as a function of position appears in the
      middle plot, and the sequence of errors in the forecast velocity
      time series after each step in the optimization appears in the
      lower plot. }
  \label{fig:fve_gun}
\end{figure}

\section{Comparing Model Classes}
\label{sec:classes}

We want to test the hypothesis that splines are better for building
models of an EOS than polynomials.  We hope to compare the bias and
variance of two sequences of estimators, one that uses splines and one
that uses polynomials.  We will study estimators for quantities of
interest that are functionals of the EOS that differ from the
functionals that correspond to the measurements.  In previous work, we
have analyzed the functionals for muzzle time and muzzle velocity for
an ideal gun, and in the sections above we have described an analysis
of velocity measurements that depend on time.  We hope that those
quantities of interest and those measurements are sufficiently
different in that the measurements are given as functions of time and
the quantities of interest are required as functions of position.

Here is the agenda:
\begin{enumerate}
\item Design two classes of EOS models, ie, pressure as a function of
  volume along an isentrope.  One class will be splines with $n$ knots
  equally spaced on a log scale, and the other will be $n^{\text{th}}$
  order polynomials.
\item Write code that draws realizations of experimental data based on
  an EOS that is outside of both model classes for any finite $n$.
\item Write code that estimates the coefficients of polynomial models
  of the EOS.
\item Calculate (or estimate with simulations) sequences (indexed by
  $n$) of bias and variance pairs of estimators for quantities of
  interest using the two classes of models.  We hope to find that the
  curve defined by the pairs for splines lies below the curve defined
  by pairs for polynomials.
\end{enumerate}

\appendix

\section{Cubic Splines}
\label{sec:splines}

Cubic splines are so named because their basis functions are piecewise
cubic.  I illustrate their basis functions for a grid of evenly spaced
knots in Fig.~\ref{fig:basis}.  Note that the second derivative is
affine between knots and that consequently if it is positive at the
knots it is positive between them.  Also if the second derivative is
positive over the domain and the first derivative is positive at the
right hand end point, the function is monotonic and convex over the
entire domain.  Finally, if the function is monotonic and positive a
the right hand end point, then it is positive over the entire domain.
\begin{figure}
  \centering
    \resizebox{\columnwidth}{!}{\includegraphics{\fig{basis}}}  
    \caption{Cubic spline basis functions and their first and second
      derivatives. The dashed plots are for $f_5$ whose knot is
      $t_5=3$.  Note that $t_k=0$ for $f_0,~f_1,~f_2$ and $f_3$, and
      $t_4=2$.}
  \label{fig:basis}
\end{figure}

Figure~\ref{fig:basis} also illustrates the symmetry of the right and
left boundaries.  The knot location $t$ of each basis function is the
point on the left where the function departs from zero.  There are
four functions at $t=0$.  There are also four functions at $t=10$, but
since their coefficients are always zero, they are not degrees of
freedom in any fit.

\subsection{Constraints}
\label{sec:constraints}

I require that the EOS be everywhere convex, monotonically decreasing
and positive.  Since I am using cubic splines, the third derivative is
piecewise constant, and for the knot positions in Fig.~\ref{fig:basis}
the constraints are equivalent to the following:
\begin{description}
\item[Second derivative positive at each knot] The matrix
  \setcounter{MaxMatrixCols}{12}
  \begin{equation}
    \label{eq:d2}
    D_2 =
    \begin{bmatrix}
      \frac{3}{2} & -\frac{5}{2} & 1     & 0   & \cdots \\ \\
      0   & \frac{2}{3} & -\frac{7}{6} & \frac{1}{2} & 0 & \cdots \\ \\
      \vdots   & 0    & \frac{3}{4} & -\frac{7}{4} & 1 & 0 & \cdots \\ \\
      & \vdots    & 0   & 1 & -2 & 1 & 0 & \cdots \\ \\
       & & & 0 & 1 & -2 & 1 & 0  \\ \\
      & & & & 0 & 1 & -2 & 1 & 0  \\ \\
      &&&&& 0 & 1 & -\frac{7}{4} &\frac{3}{4} & 0 \\ \\
      &&&&&& 0 & \frac{1}{2} & -\frac{7}{6} & \frac{2}{3} & 0 \\ \\
      &&&&&&& 0 & 1 & -\frac{5}{2} & \frac{3}{2} \\ \\
    \end{bmatrix}
  \end{equation}
  is the map from coefficients to the second derivative at the knots
  in Fig.~\ref{fig:basis}.
\item[First derivative negative at last knot] The vector
  \begin{equation*}
    D_1 = \begin{bmatrix}
    0 & \cdots & 0 & -\frac{3}{2} & \frac{3}{2}
  \end{bmatrix}
  \end{equation*}
  is the map from coefficients to the first derivative at the last knot
  in Fig.~\ref{fig:basis}.
\item[Function value positive at last knot] The vector
  \begin{equation*}
     D_0 = \begin{bmatrix}
    0 & \cdots & 0 & 1
  \end{bmatrix}
  \end{equation*}
  is the map from coefficients to the function value at the last knot
  in Fig.~\ref{fig:basis}.
\end{description}
Thus for the knots in Fig.~\ref{fig:basis}, the constraints are
\begin{align}
  G & \equiv
      \begin{bmatrix}
        - D_2 \\ D_1 \\ -D_0
      \end{bmatrix}\\
  \label{eq:constraint}
  G\cf & \preceq 0.
\end{align}

\section{Preconditioning}
\label{sec:preconditioning}

Wherever in the domain of $\eos$ there is no effect on any of the
experiments, the posterior distribution will be the same as the prior.
The domain should extend beyond the effect of the experiments.
Since the variance of the prior at $v$ is proportional $(f(v))^2$ and
$f(v) \approx \frac{C}{v^3}$,
\begin{equation}
  \label{eq:var_f}
  \sigma^2_{c}[i] \propto (v[i])^{-6},
\end{equation}
and the condition\footnote{The condition of a symmetric matrix is the
  ratio of the largest eigenvalue to the smallest eigenvalue.  See
  \eqref{eq:opt} for the calculation of $P$.} of $P$ will be
$\left( \frac{v_{\text{max}}}{v_{\text{min}}} \right)^6$.  By
changing to the variable $z$ defined as
\begin{align*}
  z[i] &\equiv v[i]^3 d[i] \\
  z &\equiv U \cdot d \text{ with} \\
  U[i,j] &= \delta_{i,j} v[i]^3,
\end{align*}
\newcommand{\UI}{U^{-1}} instead of $P$, the optimization code will use
$\UI P \UI$, which is better conditioned.  I also rescale the
constraint equation to ensure that $\left| h[i] \right|=1~\forall i$
using the matrix
\begin{equation*}
  H^{-1}[i,j] \equiv \delta_{i,j} \frac{1}{\left|h[i]\right|}
\end{equation*}


The procedure is to first solve the quadratic program specified by
\begin{align*}
  \tilde P &= \UI P \UI \\
  \tilde q &= \UI q \\
  \tilde G &= H^{-1}G \UI \\
  \tilde h &= H^{-1}h
\end{align*}
for $z$, and then derive the estimate of $d$ from
\begin{equation*}
  d = \UI z.
\end{equation*}

\section{Other Information}
\label{sec:appendix}

See other information in $\sim$/doc/HE.tex on moonlight.

\bibliographystyle{unsrt} \bibliography{local}
%
\vfill \hrule

Source file: ssh://git@xcp-stash.lanl.gov:7999/fvs/like\_eos/notes.tex

\end{document}

%%%---------------
%%% Local Variables:
%%% eval: (TeX-PDF-mode)
%%% eval: (setq ispell-personal-dictionary "./localdict")
%%% End:
